\documentclass[a4paper,12pt]{article} 
\usepackage[english,ngerman]{babel} % So ist deutsch die Standardsprache, englisch kann eingeschaltet werden. 
\usepackage[utf8]{inputenc} % Für Umlaute
\usepackage[T1]{fontenc} % Für bessere Silbentrennung
\usepackage{lmodern} % Für eine weniger pixelige Schriftart
\usepackage{graphicx} % Für Bilder (png, pdf, jpg,...)
%
%-------------Mathezeugs--------------
\usepackage{amssymb}
\usepackage{amsthm} % Für mehr Funktionen bei den theorem-Umgebungen
\usepackage{mathtools} % Eine Weiterentwicklung von amsmath
%
%-------Abstände etc.--------------
\usepackage{geometry} % Zur individuellen Anpassung der Seitenränder, z.B.:
%\geometry{top=15mm, bottom=15mm, left=20mm, right=15mm}
\usepackage{hyperref}
%
\setlength{\parindent}{0ex} % Kein Einrücken bei einem Absatz
\renewcommand{\arraystretch}{1.5}   % Zeilenbreite bei Tabellen

\renewcommand{\maketitle}{

\thispagestyle{empty}
\vspace*{5cm}


\raisebox{-5mm}{\includegraphics[width=30mm]{kit_logo_de_color.jpg}}
%
\hfill   
%
\parbox{30mm}
{ 
    \begin{flushright}
        Modul \\ Fachdidaktik 3 \\ SS 2020 
    \end{flushright}

}

\rule{\textwidth}{1pt}                                   % Linie über die ganze Textbreite
%
\begin{center}
\textbf{
Softwareprojekte im Informatikunterricht \\[1ex] % Zeilenumbruch mit Abstand
%
% Vortragsnummer und Thema
{
\Large Projektarbeit --  \glqq Ganz schön clever\grqq } \\[1ex]
% Die Befehle \glqq und \grqq stehen dabei für german left / right double quote. 
%
% Name der Vortragenden und Datum der Erstellung
%
Malte Vo\ss, \today
}
\end{center}
% 
\rule{\textwidth}{1pt}                                 % Linie über die ganze Textbreite                                  
%%%%%%%%%%%%%%%%%%%%%%%%%%%%%%%%%%%%%%%%%%%%%%%%%%%%%%%%%%%%%%%%%%%%%%%%%%%%%%%%%%%%%%%%%%%%%%%%
\vspace*{.5cm} 
\newpage

}

\newcommand{\sus}{Schülerinnen und Schüler}
\newcommand{\susn}{\sus n}

\begin{document}
    \maketitle

    %\begin{abstract}    \end{abstract} % TODO Zsf

\section{Projekte im Informatikunterricht}
    In der modernen Berufswelt gehört die Projektarbeit zur Normalität.
    Dabei werden gewünschte Produkte spezifiziert und umgesetzt.
    Zur Durchführung von Projekten gibt es Konventionen und Modelle, 
    welche die Realisierung unterstützen.
    
    Bereits im Informatikunterricht können \sus{} Projekte umsetzen
    und so Methoden und Anforderungen aus der Berufswelt kennen lernen.
    Darüber hinaus sind weitere Inhalte der Informatik Teil des Projekts 
    je nach thematischer Umsetzung.

    Durch die Projektarbeit steht das Handeln der \sus{} im Vordergrund.
    Prozessbezogene Kompetenzen werden zur Geltung gebracht.
    Insbesondere die Kooperation im Team ist eine wichtige Voraussetzung 
    für den Erfolg des Projekts.

\subsection{Verankerung im Bildungsplan}

\subsection{Voraussetzungen für Softwareprojekte}
    % OOP, GUI, Datenstrukturen

\subsection{Bewertung von Softwareprojekten}
    % Aufteilung der Arbeitslast
    % Fremdbewertung
    % Gütekriterien
    % Gewichtung von Kriterien

\section{Projekt \glqq Ganz schön clever\grqq}
    Ganz schön clever ist ein Würfelspiel von Schmidt,
    es existiert auch eine online spielbare Version 
    \footnote{\url{https://www.schmidtspiele.de/static/onlinespiele/ganz-schoen-clever/}} 
    für eine Person.
    Es besteht eine gewisse Ähnlichkeit zu Kniffel, 
    da durch den Wurf mit mehreren Würfeln Aufgaben auf einem Zettel gelöst werden.

    In der ersten Untersektion werden die Grundlagen des Spiels erläutert.
    Weiterhin werden in den folgenden Untersektionen verschiedene 
    Aspekte der Projektdurchführung beleuchtet und Vorschläge zur Umsetzung gemacht


\subsection{Grundlagen des Spiels}

% weitere subs: Projektplanung (Hefte), Entwurf, Implementierung


\end{document}