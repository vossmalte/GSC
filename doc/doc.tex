\documentclass[a4paper,12pt]{article} 
\usepackage[english,ngerman]{babel} % So ist deutsch die Standardsprache, englisch kann eingeschaltet werden. 
\usepackage[utf8]{inputenc} % Für Umlaute
\usepackage[T1]{fontenc} % Für bessere Silbentrennung
\usepackage{lmodern} % Für eine weniger pixelige Schriftart
\usepackage{graphicx} % Für Bilder (png, pdf, jpg,...)
%
%-------------Mathezeugs--------------
\usepackage{amssymb}
\usepackage{amsthm} % Für mehr Funktionen bei den theorem-Umgebungen
\usepackage{mathtools} % Eine Weiterentwicklung von amsmath
%
%-------Abstände etc.--------------
\usepackage{geometry} % Zur individuellen Anpassung der Seitenränder, z.B.:
%\geometry{top=15mm, bottom=15mm, left=20mm, right=15mm}
\usepackage{hyperref}
%
\setlength{\parindent}{0ex} % Kein Einrücken bei einem Absatz
\renewcommand{\arraystretch}{1.5}   % Zeilenbreite bei Tabellen

\renewcommand{\maketitle}{

\thispagestyle{empty}
\vspace*{5cm}


\raisebox{-5mm}{\includegraphics[width=30mm]{kit_logo_de_color.jpg}}
%
\hfill   
%
\parbox{30mm}
{ 
    \begin{flushright}
        Modul \\ Fachdidaktik 3 \\ SS 2020 
    \end{flushright}

}

\rule{\textwidth}{1pt}                                   % Linie über die ganze Textbreite
%
\begin{center}
\textbf{
Softwareprojekte im Informatikunterricht \\[1ex] % Zeilenumbruch mit Abstand
%
% Vortragsnummer und Thema
{
\Large Projektarbeit --  \glqq Ganz schön clever\grqq } \\[1ex]
% Die Befehle \glqq und \grqq stehen dabei für german left / right double quote. 
%
% Name der Vortragenden und Datum der Erstellung
%
Malte Vo\ss, \today
}
\end{center}
% 
\rule{\textwidth}{1pt}                                 % Linie über die ganze Textbreite                                  
%%%%%%%%%%%%%%%%%%%%%%%%%%%%%%%%%%%%%%%%%%%%%%%%%%%%%%%%%%%%%%%%%%%%%%%%%%%%%%%%%%%%%%%%%%%%%%%%
\vspace*{.5cm} 
\newpage

}

\newcommand{\sus}{Schülerinnen und Schüler}
\newcommand{\susn}{\sus n}

\begin{document}
    \maketitle

    %\begin{abstract}    \end{abstract} % TODO Zsf

\section{Vorwort}

\section{Projekte im Informatikunterricht}
    In der modernen Berufswelt gehört die Projektarbeit zur Normalität.
    Dabei werden gewünschte Produkte spezifiziert und umgesetzt.
    Zur Durchführung von Projekten gibt es Konventionen und Modelle, 
    welche die Realisierung unterstützen.
    
    Bereits im Informatikunterricht können \sus{} Projekte umsetzen
    und so Methoden und Anforderungen aus der Berufswelt kennen lernen.
    Darüber hinaus sind weitere Inhalte der Informatik Teil des Projekts 
    je nach thematischer Umsetzung.

    Durch die Projektarbeit steht das Handeln der \sus{} im Vordergrund.
    Prozessbezogene Kompetenzen werden zur Geltung gebracht.
    Insbesondere die Kooperation im Team ist eine wichtige Voraussetzung 
    für den Erfolg des Projekts.

\section{Verankerung im Bildungsplan}

    Der Bildungsplan 2016 für das Fach Informatik 
    \footnote{\url{http://www.bildungsplaene-bw.de/,Lde/LS/BP2016BW/ALLG/GYM/INF}}
    stellt den Rahmen für die Umsetzung des Projekts im Unterricht.

\subsection{Inhaltsbezogene Kompetenzen}
    An inhaltsbezogenen Kompetenzen stehen folgende Gesichtspunkte im Vordergrund:
    \begin{itemize}
        \item Aspekte des Projektmanagements und ein Vorgehensmodell erklären.\\
            Diese Kompetenz ist im Rahmen eines Softwareprojekts gut vermittelbar,
            da Begriffe wie Milestone, User Story, Sprint, etc. auch erlebbar sind 
            und ihre Bedeutung damit greifbarer ist.
            
        \item Entwicklung eines Prototypen mit Methoden eines Vorgehensmodells.\\
            Mit der Entwicklung eines Prototypen wird ein konkretes Ziel für das Projekt gesteckt.
            Die Verwendung der Werkzeuge eines Vorgehensmodells dient dabei als Unterstützung
            zur Umsetzung des Projekts in Teamarbeit.
        \item Anwendung erlernter Praktiken wie objektorientierte Programmierung.\\
            Im Projekt werden bereits erlernte Kompetenzen benötigt, 
            etwa beim Entwurf mit Hilfe des objektorientierten Paradigmas.
    \end{itemize}

%\subsection{Voraussetzungen für Softwareprojekte}
    % OOP, GUI, Datenstrukturen

%\subsection{Bewertung von Softwareprojekten}
    % Aufteilung der Arbeitslast
    % Fremdbewertung
    % Gütekriterien
    % Gewichtung von Kriterien

\section{Projekt \glqq Ganz schön clever\grqq}
    Ganz schön clever ist ein Würfelspiel von Schmidt,
    es existiert auch eine online spielbare Version 
    \footnote{\url{https://www.schmidtspiele.de/static/onlinespiele/ganz-schoen-clever/}} 
    für eine Person.
    Es besteht eine gewisse Ähnlichkeit zu Kniffel, 
    da durch den Wurf mit mehreren Würfeln Aufgaben auf einem Zettel gelöst werden.

    In der ersten Untersektion werden die Grundlagen des Spiels erläutert.
    Weiterhin werden in den folgenden Untersektionen verschiedene 
    Aspekte der Projektdurchführung beleuchtet und Vorschläge zur Umsetzung gemacht


\subsection{Grundlagen des Spiels}

% weitere subs: Projektplanung (Hefte), Entwurf, Implementierung


\section{Didaktische Analyse}

\subsection{Vernetzung des Bildungsplaninhalte}

    Im Softwareprojekt wird auf verschiedene inhaltliche Kompetenzen zurückgegriffen.
    Aufbauend auf bereits erlernte Kompetenzen eignet sich das 
    objektorientierte Programmierparadigma besonders zum Entwurf der Software.
    Inkarniert durch reale Objekte ergeben sich schnell Klassen für Spieler, Würfel,
    Block, Felder und Aufgaben.
    Konzepte wie Geheimnisprinzip, abstrakte Klassen und Vererbung (Aufgaben und Felder sind polymorph) 
    sind eingängig anwendbar.
    Zur Haltung von mehrer gleicher Datentypen bedarf es Datenstrukturen,
    möglich ist eine Realisierung mit Arrays, um etwa einzelne Felder in einer Aufgabe zu halten.
    Zur Berechnung der Punktzahl müssen die einzelnen Aufgaben und Felder traversiert werden.

    Wie in der Berufswelt werden weitere Werkzeuge zur Implementierung herangezogen.
    Versionierungssoftware wie git ermöglicht asynchrone Teamarbeit 
    und die Nutzung bestehender Software wie GUI-Frameworks ist grundlegend
    für die Umsetzung von Spielen.

\subsection{Didaktische Reduktion}
    % 5 Grundfragen nach Klafki (Gegenwartsbedeutung, Zukunftsbedeutung, Exemplarische Bedeutung, Thematische Strukturierung, Zugänglichkeit)
    % Reduktion komplexer Inhalte (Überschaubarkeit, Begreifbarkeit)

\subsection{Methodische Überlegungen}

    Zur Umsetzung des Softwareprojekts ist Scrum als Vorgehensmodell gewählt.
    Für \susn{} als Anfänger in der Softwareentwicklung eignet sich dieses agile Vorgehen besser als 
    traditionelle Prozessmodelle, da bereits die Festlegung der Anforderungen in Form
    von User Stories ein geringeres Abstraktionsniveau besitzt.
    Weiterhin ist durch das zyklische Modell eine Anpassung der Ziele mit jedem neuen Sprint möglich.
    Damit bietet Scrum den Entwicklern eine flexiblere Planung und verlangt weniger Einschätzung im Voraus.

    GUI-basierte Software bietet einfache Möglichkeiten zum Testen.
    Dabei gehen für Spiele Anforderungen und Testfälle meist Hand in Hand, 
    da sie von bestimmten Szenarios und Eingaben ausgehen.
    Die Überprüfung dieser Testfälle geschieht dann analog. 
    Testmethoden wie jUnit eignen sich nur in geringem Umfang, 
    da die Erzeugung von Szenarios aufwendig ist.

    Das Softwareprojekt eignet sich auch zur Umsetzung im Flipped-Classroom-Konzept.
    So können Inhalte bereits vorab per Video verfügbar gemacht werden.
    Im Unterricht kann die Lehrkraft diese Inhalte aufgreifen oder bei Schwierigkeiten helfen.
    Mit dem Flipped-Classroom-Konzept kann die Lehrkraft die Entwicklung des Projektes besser verfolgen,
    da die Unterrichtszeit durch das Handeln der \sus{} genutzt wird.

    % Dokumentation via Video

\subsection{Unterrichtsziele}

    Der Unterricht ist stark durch das Produkt am Ende des Projekts motiviert. 
    Dabei bietet die Umsetzung des Spiels große Freiheiten, 
    schon beginnend bei der Erstellung der Anforderungen an die Anwendung.
    Man sieht, dass es viele Wege zum Ziel gibt.
    Dabei ist die Festlegung des Weges ein Teil der Projektplanung

    % AA3

\section{Umsetzung des Projekts}

\subsection{Projektauftrag}
    Die Anforderungen an das Projekt sind in Form von User Stories festzuhalten.
    Beispielhaft sind hier aufsteigend grob priorisiert ein paar gegeben:
    \begin{itemize}
        \item Man kann die Würfel würfeln.
        \item Ein ausgewählter Würfel lässt sich nur in ein legales Feld eintragen.
        \begin{itemize}
            \item aufteilbar in verschiedene Farben.
        \end{itemize}
        \item Der eingetragene Würfel wird bei Seite gelegt.
        \item Ausgeschiedene Würfel (Augenzahl kleiner als eingetragener Würfel) werden aufs Silbertablett gelegt.
        \item Für jeden Spieler werden die Punkte gezählt.
        \item Joker können erspielt werden.
        \item Aktive Joker können verwendet werden.
        \item \dots
    \end{itemize}

    Diese Sammlung an User Stories ist der Product Backlog.
    
\subsection{Projektphasen}
    Zu Beginn des Softwareprojekts im Informatikunterricht lernen die \sus{} das Spiel selbst kennen.
    Daraufhin kann der Product Backlog in den einzelnen Teams erstellt werden.
    
    In den folgenden Phasen werden Sprints durchlaufen, das heißt, einzelne User Stories werden aus dem 
    Product Backlog ausgewählt und implementiert.
    Die Planung der Sprints obliegt den \susn{}, die Lehrkraft kann jedoch unterstützen und beraten,
    beispielsweise durch Einschätzung des Aufwands.
    
    Beispielsweise lassen sich folgende Sprints mit Zielen definieren:
    \begin{enumerate}
        \item Fachklassen für Würfel, Spieler, Blatt, Aufgabe, Feld implementieren
        \item GUI für die obigen Klassen erzeugen
        \item Controller implementieren
        \item Update-Methoden für die GUI hinzufügen
        \item weitere Regeln wie Runden, Joker implementieren
        \item Abschlusspräsentation
    \end{enumerate}

\subsection{Beschreibung des eigenen Prototypen}

%TODO

\section{Unterrichtliche Realisierung}

    Das Projekt ist für Gruppen von drei bis vier Personen vorgesehen.
    Der Zeitumfang beträgt im Basisfach etwa 6 Wochen.

    Ein Softwareprojekt zu benoten ist komplex, denn es wird eine Leistung über
    einen langen Zeitraum, an der mehrere \sus{} beteiligt sind, bewertet.
    Wie in vorherigen Teilen des Projekts bietet sich auch hier die Möglichkeit,
    die \sus{} mitwirken zu lassen und so transparente Kriterien zu erstellen.
    Diese Festlegung sollte bereits zu Beginn des Projektes geschehen und ist 
    als Pendant zu Abnahmekritieren zu verstehen.
    Beispielhaft kann ein Bewertungsschema so aussehen:

    \begin{itemize}
        \item[30\%] Spiel (Regeln, Hilfestellung, Spaß)
        \item[25\%] Entwurf
        \item[15\%] Dokumentation des Quellcodes
        \item[15\%] Design der GUI
        \item[15\%] Dokumentation des Softwareprojekts (Taskboards, Sprint backlogs, Testprotokolle)
    \end{itemize}

\end{document}